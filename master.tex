% Format teze zasnovan je na paketu memoir
% http://tug.ctan.org/macros/latex/contrib/memoir/memman.pdf ili
% http://texdoc.net/texmf-dist/doc/latex/memoir/memman.pdf
% 
% Prilikom zadavanja klase memoir, navedenim opcijama se podešava 
% veličina slova (12pt) i jednostrano štampanje (oneside).
% Ove parametre možete menjati samo ako pravite nezvanične verzije
% mastera za privatnu upotrebu (na primer, u b5 varijanti ima smisla 
% smanjiti 
\documentclass[12pt,oneside]{memoir} 

% Ukljuceni paketi
\usepackage{amssymb}
\usepackage{amsmath}
\usepackage{amsfonts}
\usepackage{amsthm}
\usepackage{algorithm}
\usepackage[noend]{algpseudocode}

% Teoreme, definicije
\newtheorem{theorem}{Teorema}
\newtheorem{lemma}{Lema}
\newtheorem{corrolary}{Posledica}

\theoremstyle{definition}
\newtheorem*{definition}{Definicija}

\makeatletter
\renewcommand*{\ALG@name}{Algoritam}
\makeatother

% Paket koji definiše sve specifičnosti master rada Matematičkog fakulteta
\usepackage[latinica]{matfmaster} 
%
% Podrazumevano pismo je ćirilica.
%   Ako koristite pdflatex, a ne xetex, sav latinički tekst na srpskom jeziku
%   treba biti okružen sa \lat{...} ili \begin{latinica}...\end{latinica}.
%
% Opicija [latinica]:
%   ako želite da pišete latiniciom, dodajte opciju "latinica" tj.
%   prethodni paket uključite pomoću: \usepackage[latinica]{matfmaster}.
%   Ako koristite pdflatex, a ne xetex, sav ćirilički tekst treba biti
%   okružen sa \cir{...} ili \begin{cirilica}...\end{cirilica}.
%
% Opcija [biblatex]:
%   ako želite da koristite reference na više jezika i umesto paketa
%   bibtex da koristite BibLaTeX/Biber, dodajte opciju "biblatex" tj.
%   prethodni paket uključite pomoću: \usepackage[biblatex]{matfmaster}
%
% Opcija [b5paper]:
%   ako želite da napravite verziju teze u manjem (b5) formatu, navedite
%   opciju "b5paper", tj. prethodni paket uključite pomoću: 
%   \usepackage[b5paper]{matfmaster}. Tada ima smisla razmisliti o promeni
%   veličine slova (izmenom opcije 12pt na 11pt u \documentclass{memoir}).
%
% Naravno, opcije je moguće kombinovati.
% Npr. \usepackage[b5paper,biblatex]{matfmaster}

% Pomoćni paket koji generiše nasumičan tekst u kojem se javljaju sva slova
% azbuke (nema potrebe koristiti ovo u pravim disertacijama)
% \usepackage[latinica]{pangrami}

% Datoteka sa literaturom u BibTex tj. BibLaTeX/Biber formatu
\bib{master}

% Ime kandidata na srpskom jeziku (u odabranom pismu)
\autor{Ivan Drecun}
% Naslov teze na srpskom jeziku (u odabranom pismu)
\naslov{Algoritmi za ispitivanje izomorfizma grafova}
% Godina u kojoj je teza predana komisiji
\godina{2021}
% Ime i afilijacija mentora (u odabranom pismu)
\mentor{dr Filip \textsc{Marić}, vanredni profesor\\ Univerzitet u Beogradu, Matematički fakultet}
% Ime i afilijacija prvog člana komisije (u odabranom pismu)
\komisijaA{dr Miodrag \textsc{Živković}, redovan profesor\\ Univerzitet u Beogradu, Matematički fakultet}
% Ime i afilijacija drugog člana komisije (u odabranom pismu)
\komisijaB{dr Vesna \textsc{Marniković}, docent\\ Univerzitet u Beogradu, Matematički fakultet}
% Ime i afilijacija trećeg člana komisije (opciono)
% \komisijaC{}
% Ime i afilijacija četvrtog člana komisije (opciono)
% \komisijaD{}
% Datum odbrane (odkomentarisati narednu liniju i upisati datum odbrane ako je poznat)
% \datumodbrane{}

% Apstrakt na srpskom jeziku (u odabranom pismu)
\apstr{%
	Apstrakt rada.
}

% Ključne reči na srpskom jeziku (u odabranom pismu)
\kljucnereci{ključne, reči}

\begin{document}
% ==============================================================================
% Uvodni deo teze
\frontmatter
% ==============================================================================
% Naslovna strana
\naslovna
% Strana sa podacima o mentoru i članovima komisije
\komisija
% Strana sa posvetom (u odabranom pismu)
\posveta{Mami, tati i dedi}
% Strana sa podacima o disertaciji na srpskom jeziku
\apstrakt
% Sadržaj teze
\tableofcontents*

% ==============================================================================
% Glavni deo teze
\mainmatter
% ==============================================================================

% ------------------------------------------------------------------------------
\chapter{Uvod}
% ------------------------------------------------------------------------------


% ------------------------------------------------------------------------------
\chapter{Opšti algoritam}
% ------------------------------------------------------------------------------

 U ovoj glavi predstavljeni su osnovni matematički pojmovi neophodni za dalje
 razumevanje konstrukcije opšteg algoritma za određivanje kanonske forme grafa.
 Uvedeni su pojmovi \emph{bojenja} i \emph{obojenog grafa}, nakon čega je
 prikazana konstrukcija stabla pretrage koja leži u osnovi algoritma. Nad
 stablom pogodno je definisana invarijanta koja omogućava određivanje grupe
 automorfizama grafa i kanonske forme. Na kraju su prikazani i mehanizmi
 odsecanja pretrage koji omogućavaju praktično izvršavanje algoritma u razumnom
 vremenu.

 \section{Osnovni pojmovi}

  \subsection{Obojen graf}

   \emph{Graf} $G = (V, E)$ je uređeni par konačnog \emph{skupa čvorova} $V$ i
   \emph{skupa grana} $E \subseteq {V \choose 2}$. U nastavku pretpostavljamo da
   je $V = \{1, 2, \dots, n\}$ za neki prirodan broj $n > 0$. Označimo skup svih
   grafova sa $n$ čvorova sa $\mathcal{G}_n$ (nadalje $\mathcal{G}$).

   \emph{Bojenje} grafa $G$ je surjekcija $\pi : V \to \{1, 2, \dots, k\}$ za
   neki prirodan broj $k > 0$. Označimo skup svih bojenja grafa sa $n$ čvorova
   sa $\Pi_n$ (nadalje $\Pi$).

   Broj $k$ zovemo brojem boja i označavamo ga sa $|\pi|$.  Ćelija bojenja $\pi$
   boje $c$ je skup svih čvorova te boje, odnosno $\pi^{-1}(c)$ za $c \in \{1,
   2, ..., k\}$.  Bojenje je diskretno ukoliko je $|\pi| = n$ i tada je $\pi$
   permutacija skupa $V$.

   Bojenje $\pi_1$ je finije od bojenja $\pi_2$ (u oznaci $\pi_1 \leq \pi_2$)
   ukoliko za sve $v, w \in V$ važi implikacija $\pi_2(v) < \pi_2(w) \implies
   \pi_1(v) < \pi_1(w)$.

   Označimo sa $\sim_\pi$ binarnu relaciju na skupu čvorova definisanu sa $u
   \sim_\pi v \iff \pi(u) = \pi(v)$. U pitanju je relacija ekvivalencije
   (particija) čije klase odgovaraju upravo ćelijama bojenja $\pi$.

   Particija $\alpha$ je finija od particije $\beta$ (u oznaci $\alpha \leq
   \beta$) ukoliko za sve $v, w \in V$ važi implikacija $v \alpha w \implies v
   \beta w$. Primetimo da za bojenja $\pi_1$ i $\pi_2$ važi $\pi_1 \leq \pi_2
   \implies \sim_{\pi_1} \leq \sim_{\pi_2}$, ali ne i obrnuto.

   \emph{Obojen graf} je uređeni par $(G, \pi)$ gde je $\pi$ jedno bojenje
   grafa $G$.


   \subsection{Dejstvo grupe}

   Neka je $G$ grupa i $S$ skup na kom je definisano dejstvo grupe $G$ označeno
   sa $s^g$ za $s \in S$ i $g \in G$. Orbita elementa $s$ je skup $\Omega_s^G =
   \{s^g \mid g \in G\}$.  Stabilizator elementa $s$ je skup $\Sigma_s^G = \{g
   \in G \mid s^g=s\}$ koji čini jednu podgrupu od $G$.

   Neka $S_n$ označava simetričnu grupu stepena $n$. Na skupu čvorova $V$
   definisano je dejstvo grupe sa $v^g = g(v)$ za $v \in V$ i $g \in S_n$.
   Definiciju dejstva grupe permutacija možemo proširiti i na složenije
   strukture:
   \begin{itemize}
       \item $W^g = \{w^g \mid w \in W\}$ za skup $W \subseteq V$
       \item $w^g = (v_1^g, v_2^g, \dots, v_k^g)$ za uređenu $k$-torku $w$
       \item $G^g = (V, E')$ za graf $G$ i $E' = \{e^g \mid e \in E\}$
       \item Ako je $\pi$ bojenje, $\pi^g$ je bojenje za koje važi
		   $\pi^g(v^g)=\pi(v)$ odnosno $\pi^g=\pi g^{-1}$
       \item $(G, \pi)^g = (G^g, \pi^g)$ za obojen graf $(G, \pi)$
   \end{itemize}


   \subsection{Izomorfizam}

   Obojeni grafovi $(G_1, \pi_1)$ i $(G_2, \pi_2)$ su \emph{izomorfni} (u oznaci
   $(G_1, \pi_1) \cong (G_2, \pi_2)$) ukoliko postoji $g \in S_n$ tako da je
   $(G_1, \pi_1) = (G_2, \pi_2)^g$. Takvo $g$ zovemo \emph{izomorfizam}.

   \emph{Automorfizam} obojenog grafa $(G, \pi)$ je izomorfizam tog grafa sa
   samim sobom, odnosno $g \in S_n$ za koje važi $(G, \pi) = (G, \pi)^g$. Skup
   automorfizama grafa $(G, \pi)$ označavamo sa $Aut(G, \pi)$. Zajedno sa
   operacijom kompozicije preslikavanja skup $Aut(G, \pi)$ čini \emph{grupu
   automorfizama}.


   \subsection{Kanonska forma}

   Neka je $f : \mathcal{G} \times \Pi \to S$ preslikavanje iz skupa svih
   obojenih grafova u proizvoljan skup $S$.  Kažemo da je $f$ \emph{funkcija
   invarijantna na imenovanje čvorova} ukoliko za svaki obojen graf $(G, \pi)$
   i svaku permutaciju $g \in S_n$ važi $f(G^g, \pi^g) = f(G, \pi)$.
   Neformalno, to znači da vrednost funkcije $f$ ne zavisi od konkretnog
   imenovanja čvorova grafa, već samo od njegove unutrašnje strukture.

   Ako na skupu $S$ postoji definisano dejstvo grupe $S_n$, kažemo da je $f$
   \emph{transformacija invarijantna na imenovanje čvorova} ukoliko za svaki
   obojen graf $(G, \pi)$ i svaku permutaciju $g \in S_n$ važi $f(G^g, \pi^g) =
   f(G, \pi)^g$.


   \begin{definition}
	   \emph{Kanonska forma} je preslikavanje $\mathcal{C} : \mathcal{G} \times \Pi \to
	   \mathcal{G} \times \Pi$ koje ispunjava sledeće uslove:
	   \begin{itemize}
		   \item[($\mathcal{C}1$)] Za svaki obojen graf $(G, \pi)$ važi
			   $\mathcal{C}(G, \pi) \cong (G,
			\pi)$
		\item[($\mathcal{C}2$)] $\mathcal{C}$ je funkcija invarijantna na
			imenovanje čvorova
	   \end{itemize}
   \end{definition}


 \section{Stablo pretrage}

  Označimo sa $V^*$ skup svih konačnih nizova elemenata skupa $V$. Ako je $\nu
  \in V^*$ sa $|\nu|$ označavamo dužinu niza $\nu$. Ako je $\nu = (v_1, v_2,
  \dots, v_k) \in V^*$ i $w \in V$, onda $\nu \| w$ označava niz $(v_1, v_2,
  \dots, v_k, w)$. Za $0 \leq s \leq k$ prefiks niza $\nu$ dužine $s$ označavamo
  sa $[\nu]_s = (v_1, v_2, \dots, v_s)$. Uređenje $\leq$ na skupu $V^*$
  predstavlja leksikografski poredak.

  Čvorovi stabla pretrage  predstavljeni su nizovima elemenata skupa $V$, pri
  čemu korenu stabla odgovara prazan niz. U nastavku definišemo funkcije na
  osnovu kojih ćemo definisati pravila grananja u stablu.

  \begin{definition}
   \emph{Funkcija profinjavanja} je bilo koje preslikavanje $R : \mathcal{G}
	  \times \Pi \times V^* \to \Pi$ koje za svaki obojen graf $(G, \pi)$ i
	  svako $\nu \in V^*$ zadovoljava sledeće uslove:
  
   \begin{itemize}
       \item[(R1)] $R(G, \pi, \nu) \leq \pi$
       \item[(R2)] Ako je $v \in \nu$, onda je $\{v\}$ ćelija bojenja $R(G,
     	  \pi, \nu)$
       \item[(R3)] Za svako $g \in S_n$ važi $R(G^g, \pi^g, \nu^g) = R(G,
     	 \pi, \nu)^g$
   \end{itemize}
  \end{definition}

  \begin{definition}
   \emph{Funkcija odabira ciljne ćelije} je bilo koje preslikavanje $T :
	  \mathcal{G} \times \Pi \times V^* \to \mathcal{P}(V)$ koje za svaki
	  obojen graf $(G, \pi)$ i svako $\nu \in V^*$ zadovoljava sledeće
	  uslove:
  
   \begin{itemize}
       \item[(T1)] Ako je $R(G, \pi, \nu)$ diskretno, onda je $T(G, \pi, \nu) =
     	  \emptyset$
       \item[(T2)] Ako $R(G, \pi, \nu)$ nije diskretno, onda je $T(G, \pi, \nu)$
     	  nejedinična ćelija od $R(G, \pi, \nu)$
       \item[(T3)] Za svako $g \in S_n$ važi $T(G^g, \pi^g, \nu^g) = T(G, \pi,
     	 \nu)^g$
   \end{itemize}
  \end{definition}

  Kako je graf fiksan, ove funkcije možemo smatrati funkcijama čvorova stabla.
  Funkcija profinjavanja obezbeđuje postojanje bojenja pridruženog svakom čvoru
  stabla (koje postaje finije kako se spuštamo niz stablo). Funkcija odabira
  ciljne ćelije nam omogućava da odaberemo skup čvorova grafa koji nam služi za
  konstrukciju dece tog čvora u stablu. Treći uslov u obe definicije govori da
  su u pitanju transformacije invarijantne na imenovanje čvorova.

  \begin{definition}
      \emph{Stablo pretrage} $\mathcal{T}(G, \pi)$ određeno je sledećim uslovima:
  
   \begin{itemize}
       \item[($\mathcal{T}1$)] Koren stabla $\mathcal{T}(G, \pi)$ je prazan niz $()$
       \item[($\mathcal{T}2$)] Ako je $\nu$ čvor stabla $\mathcal{T}(G, \pi)$, njegova deca u
     	stablu su $\{$\nu$ \| w \mid w \in T(G, \pi, \nu)\}$
   \end{itemize}
  \end{definition}

  Dejstvo grupe $S_n$ na stablo definiše se slično kao za bilo koju drugu
  strukturu. Naredna lema pokazuje da je ovako definisano stablo invarijantno na
  imenovanje čvorova grafa.

  \begin{lemma}
      Za svaki obojen graf $(G, \pi)$ i svako $g \in S_n$ važi
      $\mathcal{T}(G^g, \pi^g) = \mathcal{T}(G, \pi)^g$.
  \end{lemma}

  \begin{proof}
      Dokažimo da za svaki čvor $\nu$ stabla $\mathcal{T}(G, \pi)$ važi da je
      $\nu^g$ čvor stabla $\mathcal{T}(G^g, \pi^g)$. Dokaz izvodimo indukcijom
      po strukturi stabla.
      \begin{itemize}
     	 \item[] \textbf{Baza indukcije} Prazan niz je koren stabla
     		 $\mathcal{T}(G^g, \pi^g)$, pa tvrđenje trivijalno važi.
     	 \item[] \textbf{Induktivni korak} Pretpostavimo da tvrđenje važi za
     		 čvor $\nu$. Neka je $\nu \| w$ dete čvora $\nu$ za neko $w \in
     		 T(G, \pi, \nu)$. Tada je $(\nu \| w)^g = \nu^g \| w^g$, ali kako
     		 važi $w^g \in T(G, \pi, \nu)^g =_{(T3)} T(G^g, \pi^g, \nu^g)$
     		 to je $\nu^g \| w^g$ dete čvora $\nu^g$ u stablu $\mathcal{T}(G^g,
     		 \pi^g)$.
      \end{itemize}
      Time smo dokazali da je stablo $\mathcal{T}(G, \pi)^g$ podstablo od
      $\mathcal{T}(G^g, \pi^g)$ ($\mathcal{T}(G, \pi)^g \subseteq
      \mathcal{T}(G^g, \pi^g)$). Prema prethodno dokazanom važi
      $\mathcal{T}(G^g, \pi^g)^{g^{-1}} \subseteq \mathcal{T}(G, \pi)$, pa
      primenom $g$ na obe strane konačno dobijamo $\mathcal{T}(G^g, \pi^g)
      \subseteq \mathcal{T}(G, \pi)^g$.
  \end{proof}

  \begin{corrolary}
      Neka je $\nu$ čvor stabla $\mathcal{T}(G, \pi)$ i neka $\mathcal{T}(G,
      \pi, \nu)$ označava njegovo podstablo sa korenom u $\nu$. Ako je $g
      \in Aut(G, \pi)$, onda je $\nu^g$ čvor stabla $\mathcal{T}(G, \pi)$ i
      važi $\mathcal{T}(G, \pi, \nu^g) = \mathcal{T}(G, \pi, \nu)^g$.
  \end{corrolary}

  \begin{lemma}
      Neka je $\nu$ čvor stabla $\mathcal{T}(G, \pi)$ i $\pi_\nu = R(G, \pi,
      \nu)$. Tada je $Aut(G, \pi_\nu) = \Sigma_\nu^{Aut(G, \pi)}$.
  \end{lemma}

  \begin{proof}
	  Na osnovu uslova (R2) bilo koji automorfizam obojenog grafa $(G,
	  \pi_\nu)$ stabilizuje $\nu$. Sa druge strane, neka $g \in Aut(G, \pi)$
	  stabilizuje $\nu$. Tada po (R3) važi $\pi_\nu^g = R(G, \pi, \nu)^g = R(G,
	  \pi, \nu) = \pi_\nu$, pa je $g \in Aut(G, \pi_\nu)$.
  \end{proof}

 \section{Invarijanta stabla}

  \begin{definition}
	  \emph{Invarijanta stabla} je preslikavanje $\phi : \mathcal{G} \times \Pi
	  \times V^* \to F$ za neki potpuno uređen skup $F$ koje za sve obojene
	  grafove $(G, \pi)$ i različite čvorove $\nu_1, \nu_2 \in \mathcal{T}(G,
	  \pi)$ ispunjava sledeće uslove:

	  \begin{itemize}
		  \item[(\phi1)] Ako su $\nu_1, \nu_2 \in \mathcal{T}(G, \pi)$ takvi
			  da je $|\nu_1|=|\nu_2|$ i $\phi(G, \pi, \nu_1) < \phi(G,
			  \phi_0, \nu_2)$, onda za sve $\omega_1 \in \mathcal{T}(G, \pi,
			  \nu_1)$ i $\omega_2 \in \mathcal{T}(G, \pi, \nu_2)$ važi
			  $\phi(G, \pi, \omega_1) < \phi(G, \pi, \omega_2)$
		  \item[(\phi2)] Ako su $\nu_1, \nu_2 \in \mathcal{T}(G, \pi)$ takvi
			  da su $\pi_1 = R(G, \pi, \nu_1)$ i $\pi_2 = R(G, \pi, \nu_2)$
			  diskretna bojenja, onda je $\phi(G, \pi, \nu_1) = \phi(G,
			  \pi, \nu_2) \implies G^{\pi_1} = G^{\pi_2}$
		  \item[(\phi3)] $\phi$ je funkcija invarijantna na imenovanje čvorova
			  grafa
	  \end{itemize}

	  Listovi $\nu_1$ i $\nu_2$ su \emph{ekvivalentni} ako i samo ako $\phi(G,
	  \pi, \nu_1) = \phi(G, \pi, \nu_2)$.
  \end{definition}

  U nastavku ćemo kroz dve teoreme prikazati značaj ovako definisane
  invarijante stabla. Označimo za proizvoljan čvor stabla $\nu$ njegovo bojenje
  dobijeno funkcijom profinjavanja sa $\pi_\nu = R(G, \pi, \nu)$. 

  \begin{lemma}
	  Neka je $g \in Aut(G, \pi)$ i listovi $\nu_1$ i $\nu_2$ takvi da je
	  $\nu_1^g = \nu_2$. Tada su $\nu_1$ i $\nu_2$ ekvivalentni i $g =
	  \pi_{\nu_2}^{-1}\pi_{\nu_1}$.
  \end{lemma}

  \begin{proof}
	  Na osnovu svojstva ($\phi3$) invarijante stabla i činjenice da je $g$
	  automorfizam sledi $\phi(G, \pi, \nu_1) =_{(\phi_3)} \phi(G^g, \pi^g,
	  \nu_1^g) =_{g \in Aut(G, \pi)} \phi(G, \pi, \nu_1^g) = \phi(G, \pi,
	  \nu_2)$, odnosno $\nu_1$ i $\nu_2$ su ekvivalentni. Dalje, važi
	  $\pi_{\nu_2}^{-1}\pi_{\nu_1} = \pi_{\nu_1^g}^{-1}\pi_{\nu_1} =_{(R3)}
	  (\pi_{\nu_1}^g)^{-1}\pi_{\nu_1} = (\pi_{\nu_1} g^{-1})^{-1} \pi_{\nu_1} = g
	  \pi_{\nu_1}^{-1} \pi_{\nu_1} = g$.
  \end{proof}

  \begin{lemma}
	  Neka su $\alpha$ i $\beta$ diskretna bojenja finija od bojenja $\pi$.
	  Tada je $\pi^\alpha = \pi^\beta$.
  \end{lemma}

  \begin{proof}
	  Dokaz izvodimo nizom sitnih tvrđenja.

	  \begin{enumerate}
		  \item $id \leq \pi^\alpha$

			  Neka su $x$ i $y$ proizvoljni. Važi $\pi^\alpha(x) <
			  \pi^\alpha(y) \iff \pi(\alpha^{-1}(x)) < \pi(\alpha^{-1}(y))$, pa
			  kako je $\alpha \leq \pi$ sledi $\alpha(\alpha^{-1}(x)) <
			  \alpha(\alpha^{-1}(y))$ odnosno $x < y$. Kontrapozicijom dobijamo
			  i $x \leq y \implies \pi^\alpha(x) \leq \pi^\alpha(y)$.

		  \item $(\pi^\alpha)^{-1}(c) = [n, m]$ za neko $n, m \in \mathbb{N}$
			  gde je $[n, m] = \{k \in \mathbb{N} \mid n \leq k \leq m\}$

			  Za svako $x, y, z $ važi $x \leq y \leq z \implies \pi^\alpha(x)
			  \leq \pi^\alpha(y) \leq \pi^\alpha(z)$, pa ako je $\pi^\alpha(x)
			  = \pi^\alpha(z) = c$, onda je i $\pi^\alpha(y) = c$.

		  \item $\pi^\alpha(n + 1) = \pi^\alpha(n)$ ili $\pi^\alpha(n+1) =
			  \pi^\alpha(n) + 1$

			  Neka je $\pi^\alpha(n+1) \neq \pi^\alpha(n)$. Tada je $n + 1 \geq
			  n \implies \pi^\alpha(n+1) \geq \pi^\alpha(n)$, pa je
			  $\pi^\alpha(n+1) > \pi^\alpha(n)$ jer su po pretpostavci
			  različiti. Pretpostavimo da je $\pi^\alpha(n+1) > \pi^\alpha(n) +
			  1$. Tada postoji $m$ takvo da je $\pi^\alpha(m) = \pi^\alpha(n) +
			  1$, ali iz $\pi^\alpha(n) < \pi^\alpha(m) < \pi^\alpha(n+1)
			  \implies n < m < n + 1$ sledi kontradikcija.

		  \item $|(\pi^\alpha)^{-1}(c)| = |\pi^{-1}(c)|$

			  Važi da je $(\pi^\alpha)^{-1}(c) = \{m \mid \pi^\alpha(m) = c\} =
			  \{m^\alpha \mid \pi^\alpha(m^\alpha) = c\} = \{m^\alpha \mid
			  \pi(m) = c\} = \pi^{-1}(c)^\alpha$. Odatle je
			  $|(\pi^\alpha)^{-1}(c)| = |\pi^{-1}(c)^\alpha| = |\pi^{-1}(c)|$.

		  \item $(\pi^\alpha)^{-1}(c) = (\pi^\beta)^{-1}(c)$

			  Dokaz izvodimo indukcijom po $c$.

			  \begin{itemize}
				  \item[] \textbf{Baza indukcije} Kako je $1 \leq x$ za sve
					  $x$, onda iz $id \leq \pi^\alpha$ sledi $\pi^\alpha(1)
					  \leq \pi^\alpha(x)$ za sve $x$ pa je $\pi^\alpha(1) = 1$.
					  Dalje, kako je $|(\pi^\alpha)^{-1}(1)| = |\pi^{-1}(1)|$,
					  mora važiti $(\pi^\alpha)^{-1}(1) = [1, |\pi^{-1}(1)|]$.
					  Analogno se pokazuje i za $\beta$.

				  \item[] \textbf{Induktivni korak} Neka je po induktivnoj
					  pretpostavci $(\pi^\alpha)^{-1}(c) = (\pi^\beta)^{-1}(c)
					  = [n, m]$. Tada je $\pi^\alpha(m+1) \neq \pi^\alpha(m)$,
					  pa je $\pi^\alpha(m+1) = \pi^\alpha(m) + 1$ i $m+1$ je
					  najmanje u $(\pi^\alpha)^{-1}(c+1)$. Odatle važi
					  $(\pi^\alpha)^{-1}(c+1) = [m + 1, m + |\pi^{-1}(c+1)|]$.
					  Analogno se pokazuje i za $\beta$.
			  \end{itemize}
	  \end{enumerate}
  \end{proof}

  \begin{theorem}
	  Za svaki list $\nu$ važi $Aut(G, \pi) = \{\pi_{\omega}^{-1}\pi_{\nu}
	  \mid \text{$\nu$ i $\omega$ su ekvivalentni} \}$.
  \end{theorem}
  
  \begin{proof}
	  Neka je $g \in Aut(G, \pi)$. Tada je po posledici 1 $\nu^g$ list stabla
	  $\mathcal{T}(G, \pi)$. Po prethodno dokazanoj lemi 3 su $\nu$ i $\nu^g$
	  ekvivalentni i $g = \pi_{\nu^g}^{-1}\pi_\nu$ što je element skupa sa
	  desne strane jednakosti.  Sa druge strane, ako su $\nu$ i $\omega$
	  ekvivalentni, onda je $G^{\pi_\nu} = G^{\pi_\omega}$, pa je
	  $\pi_\omega^{-1}\pi_\nu \in Aut(G, \pi)$.
  \end{proof}

	Prethodna teorema pokazuje da je otkrivanjem svih čvorova ekvivalentnih
	jednom čvoru moguće odrediti grupu automorfizama datog grafa. Naravno,
	ovakav način određivanja grupe automorfizama nije veoma efikasan pošto se
	grupa generiše član po član. Ovo se može poboljšati odsecanjem pretrage o
	čemu će biti reči u narednom odeljku.

  \begin{definition}
	  Neka je $\nu^*$ list stabla $\mathcal{T}(G, \pi)$ u kom invarijanta
	  $\phi(G, \pi, \nu)$ dostiže maksimum. \emph{Kanonska forma} obojenog
	  grafa $(G, \pi)$ je funkcija $\mathcal{C}(G, \pi) = (G,
	  \pi)^{\pi_{\nu^*}}$.
  \end{definition}

  Primetimo da zbog uslova $(\phi2)$ definicija ne zavisi od izbora lista
  $\nu^*$. Naredna teorema opravdava naziv i oznaku funkcije.

  \begin{theorem}
	  Funkcija $\mathcal{C}(G, \pi)$ je kanonska forma.
  \end{theorem}

  \begin{proof}
		Dokazujemo da ovako definisana funkcija ispunjava uslove kanonske forme za
		svaki obojen graf $(G, \pi)$.
	  \begin{itemize}
		  \item [($\mathcal{C}1$)] Kako je $\mathcal{C}(G, \pi) = (G,
			  \pi)^{\pi_{\nu^*}}$ to je $\mathcal{C}(G, \pi) \cong (G, \pi)$ za
			  izomorfizam $\pi_{\nu^*}$.
		  \item [($\mathcal{C}2$)] Za svako $g \in S_n$ i svako $\nu \in
			  \mathcal{T}(G, \pi)$ važi $\nu^g \in \mathcal{T}(G, \pi)^g =
			  \mathcal{T}(G^g, \pi^g)$ kao i $\phi(G^g, \pi^g, \nu^g) = \phi(G,
			  \pi, \nu)$, pa je $\nu^*^g$ list u kom invarijanta stabla
			  $\mathcal{T}(G^g, \pi^g)$ dostiže maksimalnu vrednost.  Odatle
			  sledi $\mathcal{C}(G^g, \pi^g) = (G^g, \pi^g)^{R(G^g, \pi^g,
			  \nu^*^g)} = (G^g, \pi^g)^{\pi_{v^*}^g} = (G, \pi)^{\pi_{v^*}} =
			  \mathcal{C}(G, \pi)$ pa je $\mathcal{C}$ funkcija invarijantna na
			  imenovanje čvorova.
	  \end{itemize}
  \end{proof}

 \section{Odsecanje pretrage}

  Stablo pretrage može biti veoma veliko, pa pretraga kompletnog stabla nije
  poželjna. To možemo rešiti uvođenjem tri različite operacije odsecanja.
  \begin{itemize}
	  \item Neka su $\nu_1$ i $\nu_2$ različiti čvorovi stabla $\mathcal{T}(G,
		  \pi)$ takvi da je $|\nu_1|=|\nu_2|$ i $\phi(G, \pi, \nu_1) >
		  \phi(G, \pi, \nu_2)$. Operacija $P_A(\nu_1, \nu_2)$ podrazumeva
		  odsecanje podstabla $\mathcal{T}(G, \pi, \nu_2)$.
	  \item Neka su $\nu_1$ i $\nu_2$ različiti čvorovi stabla $\mathcal{T}(G,
		  \pi)$ takvi da je $|\nu_1|=|\nu_2|$ i $\phi(G, \pi, \nu_1) \neq
		  \phi(G, \pi, \nu_2)$. Operacija $P_B(\nu_1, \nu_2)$ podrazumeva
		  odsecanje podstabla $\mathcal{T}(G, \pi, \nu_2)$.
	  \item Neka su $\nu_1$ i $\nu_2$ različiti čvorovi stabla $\mathcal{T}(G,
		  \pi)$ takvi da je $\nu_1 < \nu_2$ i $\nu_1^g=\nu_2$ za neko $g \in
		  Aut(G, \pi)$. Operacija $P_C(\nu_1, g)$ podrazumeva odsecanje
		  podstabla $\mathcal{T}(G, \pi, \nu_2)$.
  \end{itemize}

  Naredna teorema opravdava uvođenje ovih operacija odsecanja i pokazuje da one
  ne narušavaju rezultate teorema o određivanju grupe automorfizama i kanonske
  forme iz prethodnog odeljka.

  \begin{theorem}
	  Neka je $(G, \pi)$ obojen graf.
	  \begin{enumerate}
		  \item Neka je nad stablom $\mathcal{T}(G, \pi)$ izvršen proizvoljan
			  niz operacija $P_A$ i $P_C$. Tada u dobijenom stablu postoji bar
			  jedan list $\nu$ takav da je $\phi(G, \pi, \nu) = \phi(G, \pi,
			  \nu^*)$.
		  \item Neka je $\nu_0$ list stabla $\mathcal{T}(G, \pi)$ i neka je nad
			  stablom izvršen proizvoljan niz operacija $P_B(\nu_1, \nu_2)$ i
			  $P_C$ gde je $|\nu_2| > |\nu_0|$ ili $\phi(G, \pi, \nu_2) \neq
			  \phi(G, \pi, [\nu_0]_{|\nu_2|})$ i neka su $g_1, \dots, g_k$ svi
			  automorfizmi korišćeni u izvršenim operacijama $P_C$.  Tada je
			  grupa automorfizama $Aut(G, \pi)$ generisana skupom $\{g_1,
			  \dots, g_k\} \cup \{g \in Aut(G, \pi) \mid \nu_0^g \text{ nije
			  uklonjen}\}$.
	  \end{enumerate}
  \end{theorem}

  \begin{proof}
	  Dokažimo za početak nekoliko pomoćnih tvrđenja.

	  Nijedna operacija $P_A$ ne uklanja listove u kojima je vrednost
	  invarijante maksimalna. Pretpostavimo suprotno. Neka je $\nu_1$ list u
	  kom invarijanta stabla dostiže maksimum i neka je $\nu'_1$ predak od
	  $\nu_1$. Operacija $P_A(\nu'_2, \nu'_1)$ uklanja $\nu'_1$ ako je $\phi(G,
	  \pi, \nu'_1) < \phi(G, \pi, \nu'_2)$, pa po svojstvu $(\phi1)$ za
	  proizvoljan list $\nu_2$ u $\mathcal{T}(G, \pi, \nu'_2)$ važi $\phi(G,
	  \pi, \nu_1) < \phi(G, \pi, \nu_2)$, što je u kontradikciji sa
	  pretpostavkom da je vrednost invarijante maksimalna u $\nu_1$.

	  Nijedna operacija $P_B$ ne uklanja nijedan list $\nu$ ekvivalentan listu
	  $\nu_0$ (iz drugog dela teoreme). Iz pretpostavke teoreme nijedna
	  operacija $P_B(\nu_1, \nu_2)$ ne uklanja čvor $\nu_2$ takav da je
	  $\phi(G, \pi, [\nu_0]_{|\nu_2|}) = \phi(G, \pi, \nu_2)$, pa samim tim ne
	  uklanja nijedan čvor $[\nu]_{s}$ za $0 \leq s \leq |\nu|$.

	  Nijedna operacija $P_C$ ne uklanja leksikografski najmanji među
	  ekvivalentnim listovima. Štaviše, nijedna operacija $P_C$ ne uklanja
	  leksikografski najmanji list iz $\Omega_\nu^{<g_1, \dots, g_k>}$ za bilo
	  koje $\nu$. Neka je bez umanjenja opštosti $\nu$ leksikografski najmanji
	  list u svojoj orbiti. Operacija $P_C(\omega, g)$ uklanja čvor $\omega^g$
	  ako je $\omega < \omega^g$, pa je $[\nu]_{|\omega|} \neq \omega^g$ pošto
	  je ili $[\nu]_{|\omega|} < \omega$ ili su $\omega$ i $[\nu]_{|\omega|}$ iz
	  različitih orbita.

	  \begin{enumerate}
		  \item Na osnovu dokazanih svojstava operacija $P_A$ i $P_C$ iz stabla
			  se ne uklanja leksikografski najmanji list $\nu$ ekvivalentan
			  listu $\nu^*$.
		  \item Ako je $g \in Aut(G, \pi)$, onda na osnovu dokazanih svojstva
			  operacija $P_B$ i $P_C$ važi da iz stabla nije uklonjen
			  leksikografski najmanji list oblika $\nu_0^{hg}$ za neko $h \in <
			  \negmedspace g_1, \dots, g_k \negmedspace >$, izborom $\omega =
			  \nu_0^g$. Odatle sledi da je $hg \in < \negmedspace \{g_1, \dots,
			  g_k\} \cup \{g \in Aut(G, \pi) \mid \nu_0^g \text{ nije
			  uklonjen}\} \negmedspace >$, pa je i $g$ element generisane
			  grupe.
	  \end{enumerate}
  \end{proof}

% ------------------------------------------------------------------------------
\chapter{Realizacija algoritma}
% ------------------------------------------------------------------------------

 \section{Reprezentacija podataka}

  \subsection{Permutacija}

   Permutacija $p \in S_n$ predstavljena je pomoću dva vektora. Vektor
   \texttt{pi} je definisan tako da je \texttt{pi[u] = v} onda kada je $p(u) =
   v$. Vektor \texttt{ip} definisan je analogno za inverznu permutaciju,
   odnosno \texttt{ip[u] = v} onda kada je $p^{-1}(u) = v$. {\color{red}Primer}

  \subsection{Bojenje}

   Bojenje $\pi \in \Pi_n$ predstavljeno je permutacijom \texttt{pi} i vektorom
   \texttt{cells}. Permutacija \texttt{pi} predstavlja niz koji se dobija
   nadovezivanjem ćelija $\pi^{-1}(1), \dots, \pi^{-1}(k)$ redom, pri čemu su
   elementi jedne ćelije uređeni rastuće. Preciznije, \texttt{pi} predstavlja
   permutaciju $p$ takvu da je $p(u) \in \pi^{-1}(c)$ ako i samo ako je
   $\sum_{i = 1}^{c-1} |\pi^{-1}(i)| \leq u < \sum_{i = 1}^c |\pi^{-1}(i)|$ i
   da ako važi $\pi(u) = \pi(v)$ onda je $u < v \iff p^{-1}(u) < p^{-1}(v)$.

   Vektor \texttt{cells} predstavlja niz vrednosti $cells$ dužine $n$ koji
   upotpunjava permutaciju $p$ podacima o granicama ćelija bojenja $\pi$. Ako
   pozicija $u$ predstavlja početak nove ćelije, tada je $cells_u$ takvo da ta
   ćelija obuhvata tačno pozicije $[u, cells_u)$ permutacije $p$.  Formalno,
   $$ cells_u =
     \begin{cases}
		 u + |\pi^{-1}(\pi(p(u)))|, & \quad \text{ako je } u = 1 \text{ ili }
		 \pi(p(u)) \neq \pi(p(u-1)) \\
		 -1, & \quad \text{inače.} \\
	 \end{cases}
   $$
   {\color{red}Primer}

   Primetimo da je u slučaju diskretnog bojenja $\pi$ permutacija $p$
   predstavljena sa \texttt{pi} inverz permutacije $\pi$.
   {\color{red}Primer}

   Jedna od ključnih operacija nad bojenjem je profinjavanje jedne ćelije. Neka
   je $\pi^{-1}(c)$ jedna ćelija bojenja $\pi$ i neka je svakom čvoru $v$ te
   ćelije dodeljena vrednost $t_v$. Profinjavanje ćelije podrazumeva formiranje
   bojenja $\pi' \leq \pi$ takvog da za $u$ i $v$ iz $\pi^{-1}(c)$ važi
   $\pi'(u) < \pi'(v) \iff t_u < t_v$, dok za sve ostale parove $u$ i $v$ važi
   $\pi'(u) = \pi'(v) \iff \pi(u) = \pi(v)$.

  \begin{algorithm}
	  \caption{Profinjavanje ćelije}
	  \begin{algorithmic}[]
		  \Procedure{Refine\_cell}{$G, \pi, c, t$}
		  \State{$\pi' \gets \pi$}
		  \State{$k \gets 1$}
		  \For{$t' \in $ sorted($\{t_v \mid v \in \pi^{-1}(c)\}$)}
			\State{$C_k \gets \{v \in \pi^{-1}(c) \mid t_v = t'\}$}
			\State{$\pi'(v) \gets c + k - 1$ za sve $v \in C_k$}
			\State{$k \gets k + 1$}
	      \EndFor
		  \State{$\pi'(v) \gets \pi(v) + k - 1$ za sve $v$ takve da $\pi(v) > c$ }
		  \State{$s \gets $ indeks prvog najvećeg skupa među $C_{i, 1 \leq i \leq k}$}
		  \State\Return{$\pi', C_1, \dots, C_k, s$}
		  \EndProcedure
	  \end{algorithmic}
  \end{algorithm}

  \subsection{Graf}

   {\color{red} Dodati u implementaciji podršku za obojen graf, pa dati opis.}

 \section{Funkcija profinjavanja}
  
  \begin{lemma}
	  Neka je preslikavanje $I : \Pi \times V \to \Pi$ definisano sa
	  $$
	  I(\pi, v)(w) =
	  \begin{cases}
		  \pi(w), & \quad \text{ako je } \pi(w) < \pi(v) \text{ ili } w = v \\
		  \pi(w) + 1, & \quad \text{inače}\\
	  \end{cases}
	  $$
	  i neka je $F : \mathcal{G} \times \Pi \times V^* \to \Pi$ transformacija
	  invarijantna na imenovanje čvorova takva da je $F(G, \pi, \nu) \leq \pi$.
	  Tada je preslikavanje definisano sa
	  \begin{align*}
		  $$
		  &R(G, \pi, ()) = F(G, \pi, ()) \\
		  &R(G, \pi, \nu \| w) = F(G, I(R(G, \pi, \nu), w), \nu \| w)
		  $$
	  \end{align}
	  funkcija profinjavanja.
  \end{lemma}

  \begin{proof} Pokažimo prvo nekoliko svojstava ovako definisanog
	  preslikavanja $I$.

	  (I1) Za proizvoljno bojenje $\pi$ i čvor $u$ važi da je $I(\pi, u) \leq \pi$.
	  To važi zato što ako je $\pi(v) < \pi(w)$ onda nije istovremeno $I(\pi,
	  u)(v) = \pi(v) + 1$ i $I(\pi, u)(w) = \pi(w)$ jer to povlači da je
	  $\pi(v) \geq \pi(u)$ i $\pi(w) \leq \pi(u)$, odnosno da je $\pi(w) \leq
	  \pi(v)$ što je kontradikcija. U svim ostalim slučajevima iz pretpostavke
	  sledi $I(\pi, u)(v) < I(\pi, u)(w)$.

	  (I2) $\{v\}$ je ćelija bojenja $I(\pi, v)$. Ako je $\pi(w) = \pi(v)$ i $w \neq
	  v$ onda je $I(\pi, v)(w) = I(\pi, v)(v) + 1$, a kako je $I(\pi, v) \leq
	  \pi$ onda je $\pi(v) \neq \pi(w) \iff I(\pi, v)(v) \neq I(\pi, v)(w)$ pa
	  je $I(\pi, v)(v) \neq I(\pi, v)(w)$ za sve $w \neq v$.

	  (I3) $I$ je transformacija invarijantna na imenovanje čvorova. Neka je $g \in
	  S_n$ proizvoljno. Tada je $I(\pi^g, v^g)(w^g) = \pi(w) \iff I(\pi^g,
	  v^g)(w^g) = \pi^g(w^g) \iff \pi^g(w^g) < \pi^g(v^g) \lor w^g = v^g \iff
	  \pi(w) < \pi(v) \lor w = v \iff I(\pi, v)(w) = \pi(w)$. Slično je i
	  $I(\pi^g, v^g)(w^g) = \pi(w) + 1 \iff I(\pi, v)(w) = \pi(w) + 1$, odnosno
	  $I(\pi^g, v^g)(w^g) = I(\pi, v)(w) = I(\pi, v)^g(w^g)$.

	  Dokažimo sada indukcijom da ovako definisana funkcija $R$ ispunjava
	  uslove funkcije profinjavanja.
	  \begin{itemize}
		  \item[] \textbf{Baza indukcije} 
			  \begin{itemize}
				  \item[(R1)] $R(G, \pi, ()) = F(G, \pi, ()) \leq \pi$
				  \item[(R2)] Tvrđenje trivijalno važi zato što je $()$ prazan niz
				  \item[(R3)] Za svako $g \in S_n$ važi $R(G^g, \pi^g, ()^g) = F(G^g,
					  \pi^g, ()^g) = F(G, \pi, ())^g = R(G, \pi, ())^g$
			  \end{itemize}
		  \item[] \textbf{Induktivni korak} Pretpostavimo da tvrđenje važi za $\nu$.
			  \begin{itemize}
				  \item[(R1)] $R(G, \pi, \nu \| w) = F(G, I(R(G, \pi, \nu), w),
					  \nu \| w) \leq I(R(G, \pi, \nu), w) \leq R(G, \pi, \nu)
					  \leq \pi$
				  \item[(R2)] $R(G, \pi, \nu \| w) \leq I(R(G, \pi, \nu), w)$
					  pa je ${w}$ ćelija bojenja $R(G, \pi, \nu \| w)$. Ako je
					  $v \in \nu$, onda je ${v}$ ćelija $R(G, \pi, \nu)$, pa je
					  ćelija i bojenja $R(G, \pi, \nu \| w)$ jer je $R(G, \pi,
					  \nu \| w) \leq R(G, \pi, \nu)$.
				  \item[(R3)] Za svako $g \in S_n$ važi $R(G^g, \pi^g, (\nu \|
					  w)^g) = F(G^g, I(R(G^g, \pi^g, \nu^g), w^g), (\nu \|
					  w)^g) = F(G^g, I(R(G, \pi, \nu)^g, w^g), (\nu \| w)^g) =
					  F(G^g, I(R(G, \pi, \nu), w)^g, (\nu \| w)^g) = F(G,
					  I(R(G, \pi, \nu), w), (\nu \| w))^g = R(G, \pi, \nu \|
					  w)^g$.
			  \end{itemize}
	  \end{itemize}
  \end{proof}

  Funkcija $I$ definisana u prethodnoj lemi naziva se funkcijom
  \emph{individualizacije}. Primetimo da je za funkciju $I$ moguće uzeti bilo
  koje preslikavanje koje ispunjava pokazana svojstva (I1-3).

  Kako bismo definisali konkretnu funkciju profinjavanja, potrebno je još da
  odaberemo preslikavanje $F$. U tu svrhu uvodimo pojam \emph{ekvitabilnog
  bojenja}.

  \begin{definition}
	  Označimo sa $\psi(G, u, W) = \sum_{w \in W} \psi(G, u, w)$ broj grana
	  grafa $G$ koje povezuju čvor $u$ i skup čvorova $W$.  Particija $\sim$
	  skupa čvorova $V$ je \emph{ekvitabilna} ako za svaki par čvorova $u$ i
	  $v$ takvih da je $u \sim v$ i svaku klasu $C$ particije $\sim$ važi
	  $\psi(G, u, C) = \psi(G, v, C)$.  Bojenje $\pi$ je \emph{ekvitabilno} ako
	  je particija $\sim_\pi$ ekvitabilna.
  \end{definition}

  \begin{lemma}
	  Za proizvoljno bojenje $\pi$ grafa $G$ postoji jedinstvena najgrublja
	  particija $\sim_\gamma$ koja je ekvitabilna i finija od $\sim_\pi$.
  \end{lemma}

  \begin{proof}
	  Očigledno postoji bar jedna ekvitabilna particija finija od $\sim_\pi$;
	  diskretna particija ispunjava te uslove.

	  Neka su $\sim_\alpha$ i $\sim_\beta$ dve takve particije. Definišimo
	  $\sim_\gamma$ kao tranzitivno zatvorenje unije particija $\sim_\alpha$ i
	  $\sim_\beta$, odnosno $u \sim_\gamma v \iff x_1 \sim_\cup x_2 \sim_\cup
	  \dots \sim_\cup x_k$ za neke $u = x_1, x_2, \dots, x_k = v$ pri čemu je
	  $u \sim_\cup v \iff u \sim_\alpha v \lor u \sim_\beta v$.

	  Ovako definisano $\sim_\gamma$ je relacija ekvivalencije grublja od
	  $\sim_\alpha$ i $\sim_\beta$. Pokažimo da je i ekvitabilna.

	  Neka je $u \sim_\alpha v$ i neka je $C$ proizvoljna klasa iz
	  $\sim_\gamma$. Kako je $\sim_\alpha$ finije od $\sim_\gamma$, to je $C =
	  \bigcup_{i=1}^n A_i$ za neke klase $A_1, \dots, A_n$ particije
	  $\sim_\alpha$, pa je $\psi(G, u, C) = \sum_{i=1}^n \psi(G, u, A_i)$. Kako
	  je $\alpha$ ekvitabilno, to je dalje jednako $\sum_{i=1}^n \psi(G, v,
	  A_i) = \psi(G, v, C)$. Analogno se pokazuje i za $\sim_\beta$, pa važi $u
	  \sim_\cup v \implies \psi(G, u, C) = \psi(G, v, C)$.  Konačno, ako je $u
	  \sim_\gamma v$, onda je $u = x_1 \sim_\cup x_2 \sim_\cup \dots \sim_\cup x_n = v$, pa
	  je $\psi(G, u, C) = \psi(G, x_1, C) = \psi(G, x_2, C) = \dots = \psi(G,
	  x_n, C) = \psi(G, v, C)$.
  \end{proof}
  
  Ovim smo pokazali da postoji ekvitabilno bojenje finije od $\pi$ određeno do
  na raspored ćelija. Definišimo onda funkciju $F(G, \pi, \nu)$ kao rezultat
  izvršavanja algoritma određivanja jednog takvog bojenja.

  \begin{algorithm}
	  \caption{Profinjavanje bojenja}
	  \begin{algorithmic}[]
		  \Procedure{Refine}{$G, \pi, \nu$}
		  \State {$\alpha \gets \emptyset$}
		  \If {$\nu = \nu' \| w$}
			\State {push($\alpha$, $\{w\}$)}
		  \Else
			\State {push($\alpha$, $C$) za sve $C \in \pi$}
		  \EndIf
		  \While{$\alpha \neq \emptyset$}
			\State {$W \gets $ pop($\alpha$)}
			\For{$C \in \pi$}
			  \State {$\pi, C_1, \dots, C_k, s \gets $ Refine\_cell($G, \pi, C, t_v = \psi(G, v, W)$)}
			  \If{$C \in \alpha$}
				\State {remove($\alpha$, $C$)}
				\State {push($\alpha$, $C_i$) za sve $1 \leq i \leq k$}
			  \Else
			    \State {push($\alpha$, $C_i$) za sve $1 \leq i \leq k, i \neq s$}
			  \EndIf
			\EndFor
		  \EndWhile
		  \State \Return{$\pi$}
		  \EndProcedure
	  \end{algorithmic}
  \end{algorithm}

  {\color{red} Dokaz korektnosti i svojstava za $F$ ($F \leq \pi$ i label invariant) i analiza složenosti.}

 \section{Funkcija odabira ciljne ćelije}

 \section{Invarijanta stabla}

  \begin{lemma}
	  Neka je $f : \mathcal{G} \times \Pi \times V^* \to F$ funkcija
	  invarijantna na imenovanje čvorova, pri čemu je $F$ neki potpuno uređen
	  skup i neka je $bin(G)$ binarna reprezentacija gornjeg trougla matrice
	  povezanosti grafa $G$. Tada je funkcija definisana sa
	  $$ \phi(G, \pi, \nu) =
	  \begin{cases}
		  (f(G, \pi, [\nu]_0), \dots, f(G, \pi, [\nu]_{|\nu|})), & \ \text{ako } \pi_\nu \text{ nije diskretno} \\
		  (f(G, \pi, [\nu]_0), \dots, f(G, \pi, [\nu]_{|\nu|}), bin(G^{\pi_\nu}))), & \ \text{inače}
	  \end{cases}
	  $$ invarijanta stabla pri leksikografskom poretku.
  \end{lemma}

  \begin{proof}
	  Dokažimo da tako definisana funkcija $\phi$ ispunjava uslove invarijante stabla.
	  \begin{itemize}
		  \item[(\phi1)] Za svaki čvor $\omega$ podstabla $\mathcal{T}(G, \pi,
			  \nu)$ važi da je $\phi(G, \pi, \nu) = [\phi(G, \pi,
			  \omega)]_{|\nu|}$. Neka su $\nu_1$ i $\nu_2$ čvorovi stabla takvi
			  da je $|\nu_1|=|\nu_2|$ i $\phi(G, \pi, \nu_1) < \phi(G, \pi,
			  \nu_2)$. Tada za čvorove $\omega_1 \in \mathcal{T}(G, \pi,
			  \nu_1)$ i $\omega_2 \in \mathcal{T}(G, \pi, \nu_2)$ važi
			  $[\phi(G, \pi, \omega_1)]_{|\nu_1|} < [\phi(G, \pi,
			  \omega_2)]_{|\nu_2|}$ pa je po leksikografskom poretku i $\phi(G,
			  \pi, \omega_1) < \phi(G, \pi, \omega_2)$.
		  \item[(\phi2)] Ako za listove $\nu_1$ i $\nu_2$ važi $\phi(G, \pi,
			  \nu_1) = \phi(G, \pi, \nu_2)$, onda je $bin(G^{\pi_1}) =
			  bin(G^{\pi_2})$, pa je $G^{\pi_1} = G^{\pi_2}$.
		  \item[(\phi3)] Neka je $g \in Aut(G, \pi)$ i $\nu$ čvor stabla. Tada
			  je $f(G^g, \pi^g, [\nu^g]_i) = f(G^g, \pi^g, [\nu]_i^g) = f(G,
			  \pi, [\nu]_i)$, kao i $bin((G^g)^{R(G^g, \pi^g, \nu^g)}) =
			  bin((G^g)^{\pi_\nu^g}) = bin(G^{\pi_\nu})$, pa su nizovi
			  $\phi(G^g, \pi^g, \nu^g)$ i $\phi(G, \pi, \nu)$ jednaki.
	  \end{itemize}
  \end{proof}

  Još je potrebno odabrati konkretnu funkciju $f$. Uvedimo za početak pojam
  \emph{količničkog grafa}.

  \begin{definition}
	  Neka je $(G, \pi)$ obojen graf i $\pi$ ekvitabilno. \emph{Količnički
	  graf} $Q(G, \pi) = (V_Q, d_Q, \psi_Q)$ je struktura takva da je $V_Q$
	  skup od $|\pi|$ čvorova, $d_Q : V_Q \to \mathbb{N}$ preslikavanje takvo
	  da je $d_Q(c) = |\pi^{-1}(c)|$ i $\psi_Q : V_Q^2 \to \mathbb{N}$
	  preslikavanje takvo da važi $\psi_Q(c_1, c_2) = \psi(G, v,
	  \pi^{-1}(c_2))$ za bilo koje $v \in \pi^{-1}(c_1)$.
  \end{definition}

  Primetimo da je definicija dobra zbog ekvitabilnosti bojenja $\pi$. Sledeća
  lema pokazuje značaj uvedenog pojma.

  \begin{lemma}
	  Neka je $Q : \mathcal{G}_{eq} \to \mathcal{Q}$ prethodno definisano
	  preslikavanje iz skupa obojenih grafova sa ekvitabilnim bojenjem u skup
	  količničkih grafova. $Q$ je funkcija invarijantna na imenovanje čvorova.
  \end{lemma}

  \begin{proof}
	  Neka je $g \in S_n$ proizvoljno. Pokažimo da je $Q(G^g, \pi^g) = Q(G,
	  \pi)$. Kako je $|\pi^g| = |\pi|$ to su skupovi čvorova količničkih
	  grafova jednaki. Dalje, važi $d_{Q(G^g, \pi^g)}(c) = |(\pi^g)^{-1}(c)| =
	  |\pi^{-1}(c)| = d_{Q(G, \pi)}(c)$. Konačno, neka je $u' \in
	  (\pi^g)^{-1}(c_1)$. Tada važi 
	  \begin{align*}
		  $$
		  \psi_{Q(G^g, \pi^g)}(c_1, c_2) &= \psi(G^g, u', (\pi^g)^{-1}(c_2)) \\
		  &= \sum_{v' \in (\pi^g)^{-1}(c_2)} \psi(G^g, u', v') \\
		  &= \sum_{v^g \in (\pi^g)^{-1}(c_2)} \psi(G^g, u^g, v^g) &\text{smena }u'=u^g, v'=v^g\\
		  &= \sum_{v \in \pi^{-1}(c_2)} \psi(G^g, u^g, v^g) &(\pi^g)^{-1}(c_2) = \pi^{-1}(c_2)^g\\
		  &= \sum_{v \in \pi^{-1}(c_2)} \psi(G, u, v) &\psi \text{ je invarijantno}\\
		  &= \psi(G, u, \pi^{-1}(c_2)) \\
		  &= \psi_{Q(G, \pi)}(c_1, c_2) &(\pi^g)^{-1}(c_1) = \pi^{-1}(c_1)^g\\
		  $$
	  \end{align}
	  {\color{red} Uvedi dekompoziciju za psi, pokaži da je psi invarijantno i
	  dokaži pi na -1 na g je pi na g na -1 (svojstva za psi kod grafa, a ovo
	  kod bojenja i dejstva na bojenje)}
  \end{proof}

  \begin{corrolary}
	  Preslikavanje $f_Q : \mathcal{G} \times \Pi \times V^* \to \mathcal{Q}$
	  dato sa $f_Q(G, \pi, \nu) = Q(G, R(G, \pi, \nu))$ je funkcija
	  invarijantna na imenovanje čvorova.
  \end{corrolary}

  Jasno je da za preslikavanje $f$ možemo uzeti upravo ceo količnički graf, pri
  čemu je uređenje količničkih grafova moguće realizovati uređivanjem njihovih
  binarnih reprezentacija. Ovakva definicija preslikavanja $f$ u praksi nije
  korisna zbog velike složenosti neophodne za njeno izračunavanje - potrebno je
  realizovati ceo količnički graf. Ovo je moguće rešiti posmatranjem manjeg
  dela količničkog grafa i to bez gubitka korisnih informacija.

  \begin{lemma}
	  Neka je $(G, \pi)$ obojen graf i $\nu = \nu' \| w$ čvor stabla različit od
	  korena. Uvedimo oznake $\pi_1 = R(G, \pi, \nu')$ i $\pi_2 = R(G, \pi, \nu)$.
	  Neka su $c_1, \dots, c_k$ boje takve da za sve $1 \leq i \leq k$
	  važi da $\pi_2^{-1}(c_i)$ nije ćelija bojenja $\pi_1$. Označimo sa $f_i$
	  niz vrednosti $(c_i, d_{Q(G, \pi_2)}(c_i), \psi_{Q(G, \pi_2)}(c_i, c_1),
	  \dots, \psi_{Q(G, \pi_2)}(c_i, c_k))$. Tada je preslikavanje $f(G, \pi,
	  \nu)$ čija je vrednost prazan niz $()$ u slučaju korena $\nu$, odnosno
	  dobijena nadovezivanjem nizova $f_1, \dots, f_k$ u suprotnom, funkcija
	  invarijantna na imenovanje čvorova.
  \end{lemma}

  \begin{proof}
	  Pretpostavimo da za neko $g \in S_n$ i neke boje $c$ i $d$ važi
	  $\pi_2^{-1}(c) = \pi_1^{-1}(d)$, odnosno da je $\pi_2^{-1}(c)$ ćelija
	  bojenja $\pi_1$.  Tada je $R(G^g, \pi^g, \nu^g)^{-1}(c) =
	  (\pi_2^g)^{-1}(c) = \pi_2^{-1}(c)^g = \pi_1^{-1}(d)^g = (\pi_1^g)^{-1}(d)
	  = R(G^g, \pi^g, \nu'^g)^{-1}(d)$. Analogno se pokazuje i obrnuta
	  implikacija.  Ovim smo pokazali da je niz boja $c_1, \dots, c_k$ jednak
	  za $f(G^g, \pi^g, \nu^g)$ i $f(G, \pi, \nu)$. Tvrđenje onda jednostavno
	  važi pošto je količnički graf funkcija invarijantna na imenovanje
	  čvorova.
  \end{proof}

  \begin{lemma}
	  Neka je $(G, \pi)$ obojen graf i $\nu \| w_1$ i $\nu \| w_2$ različiti
	  čvorovi stabla. Tada je $f(G, \pi, \nu \| w_1) = f(G, \pi, \nu \| w_2)$
	  ako i samo ako $f_Q(G, \pi, \nu \| w_1) = f_Q(G, \pi, \nu \| w_2)$.
  \end{lemma}
  
  \begin{proof}
	  Dokaz implikacije ulevo je trivijalan. Za smer udesno dokazaćemo
	  kontrapoziciju tvrđenja.

	  Označimo $\pi_i = R(G, \pi, \nu \| w_i)$ i $\pi_2 = R(G, \pi, \nu \|
	  w_2)$. Pretpostavimo da je $Q(G, \pi_1) \neq Q(G, \pi_2)$. Ako je
	  $V_{Q(G, \pi_1)} \neq V_{Q(G, \pi_2)}$ ili $d_{Q(G, \pi_1)} \neq d_{Q(G,
	  \pi_2)}$ onda je skup ćelija bojenja $\pi_1$ i $\pi_2$ koje nisu ćelije u
	  $\pi_\nu$ različit, pa je $f(G, \pi, \nu \| w_1) \neq f(G, \pi, \nu \|
	  w_2)$. Ako je $\psi_{Q(G, \pi_1)}(c_1, c_2) \neq \psi_{Q(G, \pi_2)}(c_1,
	  c_2)$ za neke $c_1$ i $c_2$ čije odgovarajuće ćelije iz $\pi_1$ i $\pi_2$
	  nisu ćelije bojenja $\pi_\nu$, očigledno je $f(G, \pi, \nu \| w_1) \neq
	  f(G, \pi, \nu \| w_2)$.

	  Pretpostavimo da je $\pi_1^{-1}(c_2)$ ćelija bojenja $\pi_\nu$. Tada zbog
	  ekvitabilnosti bojenja $\pi_\nu$ za proizvoljno $c_1$ važi $\psi_{Q(G,
	  \pi_1)}(c_1, c_2) = \psi_{Q(G, \pi_\nu)}(d_1, d_2)$ gde je $d_1$ takvo da
	  je $\pi_1^{-1}(c_1) \subseteq \pi_\nu^{-1}(d_1)$, a $d_2$ takvo da je
	  $\pi_1^{-1}(c_2) = \pi_\nu^{-1}(d_2)$. Odatle sledi da ne važi
	  $\psi_{Q(G, \pi_1)}(c_1, c_2) \neq \psi_{Q(G, \pi_2)}(c_1, c_2)$ jer je
	  $\psi_{Q(G, \pi_1)}(c_1, c_2) = \psi_{Q(G, \pi_\nu)}(d_1, d_2) =
	  \psi_{Q(G, \pi_2)}(c_1, c_2)$, a znamo da su u obe jednakosti isti $d_1$
	  i $d_2$ u pitanju jer su $\pi_1, \pi_2 \leq \pi_\nu$.

	  Pretpostavimo da je $\pi_1^{-1}(c_1)$ ćelija bojenja $\pi_\nu$. Tada nije
	  $\psi_{Q(G, \pi_1)}(c_1, c_2) \neq \psi_{Q(G, \pi_2)}(c_1, c_2)$ zato što
	  po prethodno pokazanom važi $|\pi_1^{-1}(c_1)|\psi_{Q(G, \pi_1)}(c_1,
	  c_2) = |\pi_1^{-1}(c_2)|\psi_{Q(G, \pi_1)}(c_2, c_1) =
	  |\pi_2^{-1}(c_2)|\psi_{Q(G, \pi_2)}(c_2, c_1) =
	  |\pi_2^{-1}(c_1)|\psi_{Q(G, \pi_2)}(c_1, c_2)$.

	  {\color{red} Dokazati svojstvo $|C_1|\psi(c_1, c_2) = |C_2|\psi(c_2, c_1)$}
  \end{proof}

 \section{Automorfizmi}

 \section{Pretraga}

 \section{Invarijanta grafa}

% ------------------------------------------------------------------------------
\chapter{Rezultati testiranja}
% ------------------------------------------------------------------------------

% ------------------------------------------------------------------------------
\chapter{Zaključak}
% ------------------------------------------------------------------------------

% ------------------------------------------------------------------------------
% Literatura
% ------------------------------------------------------------------------------
\literatura

% ==============================================================================
% Završni deo teze i prilozi
\backmatter
% ==============================================================================

% ------------------------------------------------------------------------------
% Biografija kandidata
\begin{biografija}
	Biografija.
\end{biografija}
% ------------------------------------------------------------------------------

\end{document}
