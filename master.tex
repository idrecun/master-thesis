% Format teze zasnovan je na paketu memoir
% http://tug.ctan.org/macros/latex/contrib/memoir/memman.pdf ili
% http://texdoc.net/texmf-dist/doc/latex/memoir/memman.pdf
% 
% Prilikom zadavanja klase memoir, navedenim opcijama se podešava 
% veličina slova (12pt) i jednostrano štampanje (oneside).
% Ove parametre možete menjati samo ako pravite nezvanične verzije
% mastera za privatnu upotrebu (na primer, u b5 varijanti ima smisla 
% smanjiti 
\documentclass[12pt,oneside]{memoir} 

% Ukljuceni paketi
\usepackage{amssymb}
\usepackage{amsmath}
\usepackage{amsfonts}
\usepackage{amsthm}

% Teoreme, definicije
\newtheorem{theorem}{Teorema}
\newtheorem{lemma}{Lema}
\newtheorem{corrolary}{Posledica}

\theoremstyle{definition}
\newtheorem*{definition}{Definicija}

% Paket koji definiše sve specifičnosti master rada Matematičkog fakulteta
\usepackage[latinica]{matfmaster} 
%
% Podrazumevano pismo je ćirilica.
%   Ako koristite pdflatex, a ne xetex, sav latinički tekst na srpskom jeziku
%   treba biti okružen sa \lat{...} ili \begin{latinica}...\end{latinica}.
%
% Opicija [latinica]:
%   ako želite da pišete latiniciom, dodajte opciju "latinica" tj.
%   prethodni paket uključite pomoću: \usepackage[latinica]{matfmaster}.
%   Ako koristite pdflatex, a ne xetex, sav ćirilički tekst treba biti
%   okružen sa \cir{...} ili \begin{cirilica}...\end{cirilica}.
%
% Opcija [biblatex]:
%   ako želite da koristite reference na više jezika i umesto paketa
%   bibtex da koristite BibLaTeX/Biber, dodajte opciju "biblatex" tj.
%   prethodni paket uključite pomoću: \usepackage[biblatex]{matfmaster}
%
% Opcija [b5paper]:
%   ako želite da napravite verziju teze u manjem (b5) formatu, navedite
%   opciju "b5paper", tj. prethodni paket uključite pomoću: 
%   \usepackage[b5paper]{matfmaster}. Tada ima smisla razmisliti o promeni
%   veličine slova (izmenom opcije 12pt na 11pt u \documentclass{memoir}).
%
% Naravno, opcije je moguće kombinovati.
% Npr. \usepackage[b5paper,biblatex]{matfmaster}

% Pomoćni paket koji generiše nasumičan tekst u kojem se javljaju sva slova
% azbuke (nema potrebe koristiti ovo u pravim disertacijama)
% \usepackage[latinica]{pangrami}

% Datoteka sa literaturom u BibTex tj. BibLaTeX/Biber formatu
\bib{master}

% Ime kandidata na srpskom jeziku (u odabranom pismu)
\autor{Ivan Drecun}
% Naslov teze na srpskom jeziku (u odabranom pismu)
\naslov{Algoritmi za ispitivanje izomorfizma grafova}
% Godina u kojoj je teza predana komisiji
\godina{2021}
% Ime i afilijacija mentora (u odabranom pismu)
\mentor{dr Filip \textsc{Marić}, vanredni profesor\\ Univerzitet u Beogradu, Matematički fakultet}
% Ime i afilijacija prvog člana komisije (u odabranom pismu)
\komisijaA{dr Miodrag \textsc{Živković}, redovan profesor\\ Univerzitet u Beogradu, Matematički fakultet}
% Ime i afilijacija drugog člana komisije (u odabranom pismu)
\komisijaB{dr Vesna \textsc{Marniković}, docent\\ Univerzitet u Beogradu, Matematički fakultet}
% Ime i afilijacija trećeg člana komisije (opciono)
% \komisijaC{}
% Ime i afilijacija četvrtog člana komisije (opciono)
% \komisijaD{}
% Datum odbrane (odkomentarisati narednu liniju i upisati datum odbrane ako je poznat)
% \datumodbrane{}

% Apstrakt na srpskom jeziku (u odabranom pismu)
\apstr{%
	Apstrakt rada.
}

% Ključne reči na srpskom jeziku (u odabranom pismu)
\kljucnereci{ključne, reči}

\begin{document}
% ==============================================================================
% Uvodni deo teze
\frontmatter
% ==============================================================================
% Naslovna strana
\naslovna
% Strana sa podacima o mentoru i članovima komisije
\komisija
% Strana sa posvetom (u odabranom pismu)
\posveta{Mami, tati i dedi}
% Strana sa podacima o disertaciji na srpskom jeziku
\apstrakt
% Sadržaj teze
\tableofcontents*

% ==============================================================================
% Glavni deo teze
\mainmatter
% ==============================================================================

% ------------------------------------------------------------------------------
\chapter{Uvod}
% ------------------------------------------------------------------------------


% ------------------------------------------------------------------------------
\chapter{Opšti algoritam}
% ------------------------------------------------------------------------------

 U ovoj glavi predstavljeni su osnovni matematički pojmovi neophodni za dalje
 razumevanje konstrukcije opšteg algoritma za određivanje kanonske forme grafa.
 Uvedeni su pojmovi \emph{bojenja} i \emph{obojenog grafa}, nakon čega je
 prikazana konstrukcija stabla pretrage koja leži u osnovi algoritma i na
 osnovu koje je precizno definisana kanonska forma. Prikazana je i uloga
 automorfizama u pretrazi, kao i mehanizmi za odsecanje pretrage.

 \section{Osnovni pojmovi}

  \subsection{Obojen graf}

   \emph{Graf} $G = (V, E)$ je uređeni par konačnog \emph{skupa čvorova} $V$ i
   \emph{skupa grana} $E \subseteq {V \choose 2}$. U nastavku pretpostavljamo da
   je $V = \{1, 2, \dots, n\}$ za neki prirodan broj $n > 0$. Označimo skup svih
   grafova sa $\mathcal{G}$ i skup svih grafova sa $n$ čvorova sa $\mathcal{G}_n$.

   \emph{Bojenje} grafa $G$ je surjekcija $\pi : V \to \{1, 2, \dots, k\}$ za
   neki prirodan broj $k > 0$. Označimo skup svih bojenja sa $\Pi$ i skup svih
   bojenja grafa sa $n$ čvorova sa $\Pi_n$.

   Broj $k$ zovemo brojem boja i označavamo ga sa $|\pi|$.  Ćelija bojenja $\pi$
   boje $c$ je skup svih čvorova te boje, odnosno $\pi^{-1}(c)$ za $c \in \{1,
   2, ..., k\}$.  Bojenje je diskretno ukoliko je $|\pi| = n$ i tada je $\pi$
   permutacija skupa $V$.

   Bojenje $\pi_1$ je finije od bojenja $\pi_2$ (u oznaci $\pi_1 \leq \pi_2$)
   ukoliko za sve $v, w \in V$ važi implikacija $\pi_2(v) < \pi_2(w) \implies
   \pi_1(v) < \pi_1(w)$.

   \emph{Obojen graf} je uređeni par $(G, \pi)$ gde je $\pi$ jedno bojenje
   grafa $G$.


   \subsection{Dejstvo grupe $S_n$}

   Neka $S_n$ označava simetričnu grupu stepena $n$. Sliku čvora $v \in V$ pod
   permutacijom $g \in S_n$ označavamo sa $v^g$. Ovim je definisano jedno
   dejstvo grupe $S_n$ na skup $V$. Orbita čvora $v$ pod tim dejstvom je skup
   $\Omega_v = \{ v^g \mid g \in S_n \}$. Stabilizator čvora $v$ je skup
   $\Sigma_v = \{ g \in S_n \mid v^g = v \}$ koji čini jednu podgrupu od $S_n$.
   Definiciju dejstva grupe permutacija možemo proširiti i na složenije
   strukture:
   \begin{itemize}
       \item $W^g = \{w^g \mid w \in W\}$ za skup $W \subseteq V$
       \item $w^g = (v_1^g, v_2^g, \dots, v_k^g)$ za uređenu $k$-torku $w$
       \item $G^g = (V, E')$ za graf $G$ i $E' = \{e^g \mid e \in E\}$
       \item Ako je $\pi$ bojenje, $\pi^g$ je bojenje za koje važi
		   $\pi^g(v^g)=\pi(v)$ odnosno $\pi^g=\pi g^{-1}$
       \item $(G, \pi)^g = (G^g, \pi^g)$ za obojen graf $(G, \pi)$
   \end{itemize}


   \subsection{Izomorfizam}

   Obojeni grafovi $(G_1, \pi_1)$ i $(G_2, \pi_2)$ su \emph{izomorfni} (u oznaci
   $(G_1, \pi_1) \cong (G_2, \pi_2)$) ukoliko postoji $g \in S_n$ tako da je
   $(G_1, \pi_1) = (G_2, \pi_2)^g$. Takvo $g$ zovemo \emph{izomorfizam}.

   \emph{Automorfizam} obojenog grafa $(G, \pi)$ je izomorfizam tog grafa sa
   samim sobom, odnosno $g \in S_n$ za koje važi $(G, \pi) = (G, \pi)^g$. Skup
   automorfizama grafa $(G, \pi)$ označavamo sa $Aut(G, \pi)$. Zajedno sa
   operacijom kompozicije preslikavanja skup $Aut(G, \pi)$ čini \emph{grupu
   automorfizama}.


   \subsection{Kanonska forma}

   Neka je $f : \mathcal{G} \times \Pi \to S$ preslikavanje iz skupa svih
   obojenih grafova u proizvoljan skup $S$.  Kažemo da je $f$ invarijantno na
   imenovanje čvorova ukoliko za svaki obojen graf $(G, \pi)$ i svaku
   permutaciju $g \in S_n$ važi $f(G^g, \pi^g) = f(G, \pi)$. Neformalno, to
   znači da vrednost funkcije $f$ ne zavisi od konkretnog imenovanja čvorova
   grafa, već samo od njegove unutrašnje strukture. \\
   \\
   \emph{Kanonska forma} je funkcija $\mathcal{C} : \mathcal{G} \times \Pi \to
   \mathcal{G} \times \Pi$ koja ispunjava sledeće uslove:
   \begin{itemize}
       \item[($\mathcal{C}1$)] Za svaki obojen graf $(G, \pi)$ važi $\mathcal{C}(G, \pi) \cong (G,
        \pi)$
    \item[($\mathcal{C}2$)] $\mathcal{C}$ je invarijantno na imenovanje čvorova
   \end{itemize}


 \section{Stablo pretrage}

  Označimo sa $V^*$ skup svih konačnih nizova elemenata skupa $V$. Ako je $\nu
  \in V^*$ sa $|\nu|$ označavamo dužinu niza $\nu$. Ako je $\nu = (v_1, v_2,
  \dots, v_k) \in V^*$ i $w \in V$, onda $\nu \| w$ označava niz $(v_1, v_2,
  \dots, v_k, w)$. Za $0 \leq s \leq k$ prefiks niza $\nu$ dužine $s$ označavamo
  sa $[\nu]_s = (v_1, v_2, \dots, v_s)$. Uređenje $\leq$ na skupu $V^*$
  predstavlja leksikografski poredak.

  Čvorovi stabla pretrage  predstavljeni su nizovima elemenata skupa $V$, pri
  čemu korenu stabla odgovara prazan niz. U nastavku definišemo funkcije na
  osnovu kojih ćemo definisati pravila grananja u stablu.

  \begin{definition}
   \emph{Funkcija profinjavanja} je bilo koja funkcija $R : \mathcal{G} \times
      \Pi \times V^* \to \Pi$ koja za svaki obojen graf $(G, \pi_0)$ i svako
      $\nu \in V^*$ zadovoljava sledeće uslove:
  
   \begin{itemize}
       \item[(R1)] $R(G, \pi_0, \nu) \leq \pi_0$
       \item[(R2)] Ako je $v \in \nu$, onda je $\{v\}$ ćelija bojenja $R(G,
     	  \pi_0, \nu)$
       \item[(R3)] Za svako $g \in S_n$ važi $R(G^g, \pi_0^g, \nu^g) = R(G,
     	 \pi_0, \nu)^g$
   \end{itemize}
  \end{definition}

  \begin{definition}
   \emph{Funkcija odabira ciljne ćelije} je bilo koja funkcija $T : \mathcal{G}
      \times \Pi \times V^* \to \mathcal{P}(V)$ koja za svaki obojen graf $(G,
      \pi_0)$ i svako $\nu \in V^*$ zadovoljava sledeće uslove:
  
   \begin{itemize}
       \item[(T1)] Ako je $R(G, \pi_0, \nu)$ diskretno, onda je $T(G, \pi_0, \nu) =
     	  \emptyset$
       \item[(T2)] Ako $R(G, \pi_0, \nu)$ nije diskretno, onda je $T(G, \pi_0, \nu)$
     	  nejedinična ćelija od $R(G, \pi_0, \nu)$
       \item[(T3)] Za svako $g \in S_n$ važi $T(G^g, \pi_0^g, \nu^g) = T(G, \pi_0,
     	 \nu)^g$
   \end{itemize}
  \end{definition}

  Kako je graf fiksan, ove funkcije možemo smatrati funkcijama čvorova stabla.
  Funkcija profinjavanja obezbeđuje postojanje bojenja pridruženog svakom čvoru
  stabla (koje postaje finije kako se spuštamo niz stablo). Funkcija odabira
  ciljne ćelije nam omogućava da odaberemo skup čvorova grafa koji nam služi za
  konstrukciju dece tog čvora u stablu. Treći uslov u obe definicije je
  varijanta pomenute invarijantnosti na imenovanje čvorova. On govori da
  funkcije treba da zavise samo od unutrašnje strukture grafa i da
  transformacije koje one vrše ne zavise od imenovanja čvorova grafa.

  \begin{definition}
      \emph{Stablo pretrage} $\mathcal{T}(G, \pi_0)$ određeno je sledećim uslovima:
  
   \begin{itemize}
       \item[($\mathcal{T}1$)] Koren stabla $\mathcal{T}(G, \pi_0)$ je prazan niz $()$
       \item[($\mathcal{T}2$)] Ako je $\nu$ čvor stabla $\mathcal{T}(G, \pi_0)$, njegova deca u
     	stablu su $\{$\nu$ \| w \mid w \in T(G, \pi_0, \nu)\}$
   \end{itemize}
  \end{definition}

  % Iz definicije je jasno da je čvor stabla $\nu$ list ako i samo ako je bojenje
  % $R(G, \pi_0, \nu)$ diskretno.

  Dejstvo grupe $S_n$ na stablo definiše se slično kao za bilo koju drugu
  strukturu. Naredna lema pokazuje da je ovako definisano stablo invarijantno na
  imenovanje čvorova grafa.

  \begin{lemma}
      Za svaki obojen graf $(G, \pi_0)$ i svako $g \in S_n$ važi
      $\mathcal{T}(G^g, \pi_0^g) = \mathcal{T}(G, \pi_0)^g$.
  \end{lemma}

  \begin{proof}
      Dokažimo da za svaki čvor $\nu$ stabla $\mathcal{T}(G, \pi_0)$ važi da je
      $\nu^g$ čvor stabla $\mathcal{T}(G^g, \pi_0^g)$. Dokaz izvodimo indukcijom
      po strukturi stabla.
      \begin{itemize}
     	 \item[] \textbf{Baza indukcije} Prazan niz je koren stabla
     		 $\mathcal{T}(G^g, \pi_0^g)$, pa tvrđenje trivijalno važi.
     	 \item[] \textbf{Induktivni korak} Pretpostavimo da tvrđenje važi za
     		 čvor $\nu$. Neka je $\nu \| w$ dete čvora $\nu$ za neko $w \in
     		 T(G, \pi_0, \nu)$. Tada je $(\nu \| w)^g = \nu^g \| w^g$, ali kako
     		 važi $w^g \in T(G, \pi_0, \nu)^g =_{(T3)} T(G^g, \pi_0^g, \nu^g)$
     		 to je $\nu^g \| w^g$ dete čvora $\nu^g$ u stablu $\mathcal{T}(G^g,
     		 \pi_0^g)$.
      \end{itemize}
      Time smo dokazali da je stablo $\mathcal{T}(G, \pi_0)^g$ podstablo od
      $\mathcal{T}(G^g, \pi_0^g)$ ($\mathcal{T}(G, \pi_0)^g \subseteq
      \mathcal{T}(G^g, \pi_0^g)$). Prema prethodno dokazanom važi
      $\mathcal{T}(G^g, \pi_0^g)^{g^{-1}} \subseteq \mathcal{T}(G, \pi_0)$, pa
      primenom $g$ na obe strane konačno dobijamo $\mathcal{T}(G^g, \pi_0^g)
      \subseteq \mathcal{T}(G, \pi_0)^g$.
  \end{proof}

  \begin{corrolary}
      Neka je $\nu$ čvor stabla $\mathcal{T}(G, \pi_0)$ i neka $\mathcal{T}(G,
      \pi_0, \nu)$ označava njegovo podstablo sa korenom u $\nu$. Ako je $g
      \in Aut(G, \pi_0)$, onda je $\nu^g$ čvor stabla $\mathcal{T}(G, \pi)$ i
      važi $\mathcal{T}(G, \pi_0, \nu^g) = \mathcal{T}(G, \pi_0, \nu)^g$.
  \end{corrolary}

  \begin{lemma}
      Neka je $\nu$ čvor stabla $\mathcal{T}(G, \pi_0)$ i $\pi = R(G, \pi_0,
      \nu)$. Tada je $Aut(G, \pi) = \Sigma_\nu^{Aut(G, \pi_0)}$.
  \end{lemma}

  \begin{proof}
      Na osnovu uslova (R2) bilo koji automorfizam obojenog grafa $(G, \pi)$
      stabilizuje $\nu$. Sa druge strane, neka $g \in Aut(G, \pi_0)$ stabilizuje
      $\nu$. Tada po (R3) važi $\pi^g = R(G, \pi_0, \nu)^g = R(G, \pi_0, \nu) =
      \pi$, pa je $g \in Aut(G, \pi)$.
  \end{proof}

 \section{Invarijanta stabla}

  \begin{definition}
	  \emph{Invarijanta stabla} je funkcija $\phi : \mathcal{G} \times \Pi
	  \times V^* \to F$ za neki potpuno uređen skup $F$ koji za sve obojene
	  grafove $(G, \pi_0)$ i različite čvorove $\nu_1, \nu_2 \in \mathcal{T}(G,
	  \pi_0)$ ispunjava sledeće uslove:

	  \begin{itemize}
		  \item[(\phi1)] Ako su $\nu_1, \nu_2 \in \mathcal{T}(G, \pi_0)$ takvi
			  da je $|\nu_1|=|\nu_2|$ i $\phi(G, \pi_0, \nu_1) < \phi(G,
			  \phi_0, \nu_2)$, onda za sve $\omega_1 \in \mathcal{T}(G, \pi_0,
			  \nu_1)$ i $\omega_2 \in \mathcal{T}(G, \pi_0, \nu_2)$ važi
			  $\phi(G, \pi_0, \omega_1) < \phi(G, \pi_0, \omega_2)$
		  \item[(\phi2)] Ako su $\nu_1, \nu_2 \in \mathcal{T}(G, \pi_0)$ takvi
			  da su $\pi_1 = R(G, \pi_0, \nu_1)$ i $\pi_2 = R(G, \pi_0, \nu_2)$
			  diskretna bojenja, onda je $\phi(G, \pi_0, \nu_1) = \phi(G,
			  \pi_0, \nu_2) \iff G^{\pi_1} = G^{\pi_2}$
		  \item[(\phi3)] $\phi$ je invarijantno na imenovanje čvorova grafa
	  \end{itemize}

	  Listovi $\nu_1$ i $\nu_2$ su \emph{ekvivalentni} ako i samo ako $\phi(G,
	  \pi_0, \nu_1) = \phi(G, \pi_0, \nu_2)$.
  \end{definition}

  U nastavku ćemo kroz dve teoreme prikazati značaj ovako definisane
  invarijante stabla. Označimo za proizvoljan čvor stabla $\nu$ njegovo bojenje
  dobijeno funkcijom profinjavanja sa $\pi_\nu = R(G, \pi_0, \nu)$. 

  \begin{lemma}
	  Neka je $g \in Aut(G, \pi_0)$ i listovi $\nu_1$ i $\nu_2$ takvi da je
	  $\nu_1^g = \nu_2$. Tada su $\nu_1$ i $\nu_2$ ekvivalentni i $g =
	  \pi_{\nu_2}^{-1}\pi_{\nu_1}$.
  \end{lemma}

  \begin{proof}
	  Na osnovu svojstva ($\phi3$) invarijante stabla i činjenice da je $g$
	  automorfizam sledi $\phi(G, \pi_0, \nu_1) =_{(\phi_3)} \phi(G^g, \pi_0^g,
	  \nu_1^g) =_{g \in Aut(G, \pi_0)} \phi(G, \pi_0, \nu_1^g) = \phi(G, \pi_0,
	  \nu_2)$, odnosno $\nu_1$ i $\nu_2$ su ekvivalentni. Dalje, važi
	  $\pi_{\nu_2}^{-1}\pi_{\nu_1} = \pi_{\nu_1^g}^{-1}\pi_{\nu_1} =_{(R3)}
	  (\pi_{\nu_1}^g)^{-1}\pi_{\nu_1} = (\pi_{\nu_1} g^{-1})^{-1} \pi_{\nu_1} = g
	  \pi_{\nu_1}^{-1} \pi_{\nu_1} = g$.
  \end{proof}

  \begin{theorem}
	  Za svaki list $\nu$ važi $Aut(G, \pi_0) = \{\pi_{\omega}^{-1}\pi_{\nu}
	  \mid \text{$\nu$ i $\omega$ su ekvivalentni} \}$.
  \end{theorem}
  
  \begin{proof}
	  Neka je $g \in Aut(G, \pi_0)$. Tada je po posledici 1 $\nu^g$ list stabla
	  $\mathcal{T}(G, \pi_0)$. Po prethodno dokazanoj lemi 3 su $\nu$ i $\nu^g$
	  ekvivalentni i $g = \pi_{\nu^g}^{-1}\pi_\nu$ što je element desne strane.
	  Sa druge strane, ako su $\nu$ i $\omega$ ekvivalentni, onda je
	  $G^{\pi_\nu} = G^{\pi_\omega}$, pa je $\pi_\omega^{-1}\pi_\nu \in Aut(G,
	  \pi_0)$.
  \end{proof}

	Prethodna teorema pokazuje da je otkrivanjem svih čvorova ekvivalentnih
	jednom čvoru moguće odrediti grupu automorfizama datog grafa. Naravno, ovakav
	način određivanja grupe automorfizama nije veoma efikasan pošto se u ovom
	slučaju grupa generiše član po član. Ovo se može poboljšati odsecanjem
	pretrage o čemu će biti reči u narednom odeljku.

	Druga teorema tiče se mogućnosti korektnog definisanja kanonske forme na
	osnovu invarijante stabla.

  \begin{definition}
	  Neka je $\nu^*$ list stabla $\mathcal{T}(G, \pi_0)$ u kom invarijanta
	  $\phi(G, \pi_0, \nu)$ dostiže maksimum. \emph{Kanonska forma} obojenog
	  grafa $(G, \pi_0)$ je funkcija $\mathcal{C}(G, \pi_0) = (G,
	  \pi_0)^{\pi_{\nu^*}}$.
  \end{definition}

  Primetimo da zbog uslova $(\phi2)$ definicija ne zavisi od izbora lista
  $\nu^*$. Naredna teorema opravdava naziv i oznaku funkcije.

  \begin{theorem}
	  Funkcija $\mathcal{C}(G, \pi_0)$ je kanonska forma.
  \end{theorem}

  \begin{proof}
		Dokazujemo da ovako definisana funkcija ispunjava uslove kanonske forme za
		svaki obojen graf $(G, \pi)$.
	  \begin{itemize}
			\item [] ($\mathcal{C}1$) Kako je $\mathcal{C}(G, \pi) = (G,
					\pi)^{\pi_{\nu^*}}$ to je $\mathcal{C}(G, \pi) \cong (G, \pi)$ za
			izomorfizam $\pi_{\nu^*}$.
			\item [] ($\mathcal{C}2$) Za svako $g \in S_n$ i svako $\nu \in
			\mathcal{T}(G, \pi)$ važi $\nu^g \in \mathcal{T}(G, \pi)^g =
			\mathcal{T}(G^g, \pi^g)$ kao i $\phi(G^g, \pi^g, \nu^g) = \phi(G, \pi,
					\nu)$, pa je $\nu^*^g$ čvor u kom invarijanta stabla
			$\mathcal{T}(G^g, \pi^g)$ dostiže maksimalnu vrednost. Odatle sledi
			$\mathcal{C}(G^g, \pi^g) = (G^g, \pi^g)^{R(G^g, \pi^g, \nu^*^g)} = (G,
					\pi)^g^{\pi_{v^*}^g} = (G, \pi)^{\pi_{v^*}} = \mathcal{C}(G, \pi)$ pa
			je $\mathcal{C}$ invarijantno na imenovanje čvorova.
	  \end{itemize}
  \end{proof}

 \section{Odsecanje pretrage}

% ------------------------------------------------------------------------------
\chapter{Realizacija algoritma}
% ------------------------------------------------------------------------------

% ------------------------------------------------------------------------------
\chapter{Rezultati testiranja}
% ------------------------------------------------------------------------------

% ------------------------------------------------------------------------------
\chapter{Zaključak}
% ------------------------------------------------------------------------------

% ------------------------------------------------------------------------------
% Literatura
% ------------------------------------------------------------------------------
\literatura

% ==============================================================================
% Završni deo teze i prilozi
\backmatter
% ==============================================================================

% ------------------------------------------------------------------------------
% Biografija kandidata
\begin{biografija}
	Biografija.
\end{biografija}
% ------------------------------------------------------------------------------

\end{document}
